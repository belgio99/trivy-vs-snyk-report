\chapter{Confronto}
\section{Scansione delle vulnerabilità}
Il database di entrambi gli strumenti è un aspetto cruciale per la qualità e l'accuratezza delle analisi di vulnerabilità. Trivy e Snyk utilizzano database di vulnerabilità differenti, che influenzano la loro capacità di identificare e classificare le vulnerabilità.


In questa parte di testing, si è voluto valutare la capacità di Trivy e Snyk di identificare e classificare le vulnerabilità in base al database utilizzato. Per fare ciò, si è utilizzato un campione di immagini di container e repository Git, contenenti vulnerabilità note e ben documentate, e si è confrontato il risultato delle scansioni di Trivy e Snyk con le informazioni disponibili nei database VulnDB e Snyk. I risultati di questo test sono stati valutati in base alla precisione, alla completezza e alla tempestività delle informazioni fornite da Trivy e Snyk, e alla loro capacità di identificare e classificare le vulnerabilità in base al database utilizzato.
L'aggiornamento dei database avviene per ogni tool in modo differente. Trivy, infatti, controlla la presenza di aggiornamenti del database ad ogni esecuzione, e scarica automaticamente la versione più recente. Snyk, invece, offre la possibilità di aggiornare manualmente il database, ma non è possibile controllare la presenza di aggiornamenti in modo automatico.

Le immagini Docker sottoposte a scansione sono state selezionate in base alla loro popolarità e alla loro rilevanza nel panorama delle applicazioni e dei servizi cloud. In particolare, si è scelto di testare le seguenti immagini:
\begin{itemize}
   \item \textbf{nginx}: uno dei principali server web e reverse proxy.
   \item \textbf{mongo}: database NoSQL flessibile e scalabile basato su MongoDB.
   \item \textbf{wordpress}: piattaforma di blogging e CMS.
   \item \textbf{alpine}: una delle immagini di container più leggere e minimali, basata su Alpine Linux. È ampiamente usata come base per altre immagini di container.
   \item \textbf{node}: ambiente di esecuzione per JavaScript basato su Chrome V8.
\end{itemize}
In particolare, si è voluto testare due tipi di versioni: la versione più recente, e una versione più datata. Nella tabella \ref{tab:scan_results} sono riportati i risultati ottenuti da entrambi i tool per ciascuna immagine testata.
\begin{table}[H]
   \centering
   \begin{tabularx}{\textwidth}{|l|l|X|X|X|X|X|X|X|X|}
      \hline
      \textbf{Immagine}          & \textbf{Vers.} & \multicolumn{4}{c|}{\textbf{Snyk}} & \multicolumn{4}{c|}{\textbf{Trivy}}                                                                                              \\ \cline{3-10}
                                 &                & \textbf{Crit.}                     & \textbf{High}                       & \textbf{Mid} & \textbf{Low} & \textbf{Crit.} & \textbf{High} & \textbf{Mid} & \textbf{Low} \\ \hline
      \multirow{2}{*}{nginx}     & latest         & 1                                  & 6                                   & 3            & 78           & 2              & 16            & 34           & 83           \\ \cline{2-10}
                                 & v1.23.0        & 10                                 & 39                                  & 73           & 111          & 16             & 80            & 139          & 115          \\ \hline
      \multirow{2}{*}{mongo}     & latest         & 0                                  & 0                                   & 1            & 12           & 0              & 0             & 1            & 21           \\ \cline{2-10}
                                 & v4.4.3         & 0                                  & 7                                   & 88           & 69           & 0              & 13            & 195          & 105          \\ \hline
      \multirow{2}{*}{wordpress} & latest         & 1                                  & 1                                   & 3            & 150          & 3              & 50            & 143          & 326          \\ \cline{2-10}
                                 & v6.0.0         & 23                                 & 66                                  & 142          & 228          & 49             & 340           & 509          & 547          \\ \hline
      \multirow{2}{*}{alpine}    & latest         & 0                                  & 0                                   & 0            & 0            & 0              & 0             & 0            & 0            \\ \cline{2-10}
                                 & v3.11          & 1                                  & 0                                   & 0            & 0            & 1              & 0             & 0            & 0            \\ \hline
      \multirow{2}{*}{node}      & latest         & 1                                  & 4                                   & 3            & 160          & 5              & 73            & 246          & 481          \\ \cline{2-10}
                                 & v16.0.0        & 48                                 & 183                                 & 275          & 390          & 140            & 970           & 1353         & 1417         \\ \hline
   \end{tabularx}
   \caption{Risultati delle scansioni di Snyk e Trivy per ciascuna immagine testata.}
   \label{tab:scan_results}
\end{table}
\subsection{Risultati delle scansioni}
Dai risultati ottenuti, si evincono le seguenti considerazioni:
\subsubsection{Numero di vulnerabilità rilevate}
Trivy generalmente avvisa di un numero maggiore di vulnerabilità rispetto a Snyk. Questo è particolarmente evidente per le versioni più datate delle immagini, dove Trivy rileva un numero significativamente maggiore di vulnerabilità. Da solo, il mero numero di vulnerabilità rilevate non è un indicatore della qualità o dell'accuratezza delle scansioni, per via della presenza di eventuali falsi positivi o di vulnerabilità non rilevate.


\subsubsection{Differenza nella classificazione delle stesse vulnerabilità}
Gli strumenti hanno riportato, in molti casi, una differenza nelle classificazioni delle stesse vulnerabilità. Questo è dovuto alla differenza di valutazione delle vulnerabilità da parte dei database utilizzati. In particolare, Snyk tende a classificare le vulnerabilità in modo più conservativo, assegnando un numero inferiore di vulnerabilità di livello critico e alto rispetto a Trivy. Ad esempio, per l'immagine \texttt{nginx:latest}, entrambi i tool hanno rilevato correttamente la vulnerabilità CVE-2023-6879, relativa ad un buffer overflow presente nella libreria \texttt{AOMedia}, ma mentre Trivy ha classificato la vulnerabilità come di livello critico, Snyk ha classificato la stessa come di livello basso. Questa differenza è dovuta alle sorgenti di informazioni di valutazione utilizzate dai due tool:
\begin{itemize}
   \item Trivy utilizza la classificazione NVD, che è nota per essere meno conservativa nella classificazione delle vulnerabilità, assegnando quindi un numero maggiore di vulnerabilità di livello critico o alto.
   \item Snyk invece assegna la categoria in base a tre elementi di valutazione:
         \begin{itemize}
            \item L'analisi interna condotta dal team di Snyk.
            \item Una valutazione della gravità fornita dal team di sicurezza del manutentore della distribuzione Linux.
            \item La gravità della vulnerabilità secondo il database NVD.
         \end{itemize}
\end{itemize}
Visitando infatti il sito web \texttt{security-tracker.debian.org}, è possibile osservare che la vulnerabilità CVE-2023-6879 è stata classificata come di livello basso dal team di sicurezza di Debian, e questo ha influenzato in modo predominante la classificazione di Snyk.
\subsection{Formati di output}
Entrambi gli strumenti offrono la possibilità di generare report in vari formati per permetterne, ad esempio, l'integrazione diretta con altri strumenti.
\subsubsection{Trivy}
Il formato di output di default di Trivy è il formato tabulare (Figura \ref{fig:trivy_output_fmt}), che fornisce un report strutturato e facilmente leggibile delle vulnerabilità rilevate.
\begin{figure}[H]
   \centering
   \includegraphics[width=1\textwidth]{immagini/capitolo2/trivy_output_fmt.png}
   \caption{Formato di output di default di Trivy}
   \label{fig:trivy_output_fmt}
\end{figure}
Inoltre, Trivy permette di generare report nei seguenti formati:
\begin{itemize}
   \item \textbf{JSON}
   \item \textbf{JUnit XML}: un formato di output standard per i risultati dei test.
   \item \textbf{Sarif}: un formato di output standard per gli strumenti di analisi statica del codice.
   \item \textbf{Personalizzato}: se i formati predefiniti non soddisfano le esigenze, è possibile specificare un formato di output completamente personalizzato.
\end{itemize}

\subsubsection{Snyk}
Nel formato di default, Snyk restituisce un report in formato "lista" (Figura \ref{fig:snyk_output_fmt}), con l'elenco di tutte le vulnerabilità rilevate.
\begin{figure}[H]
   \centering
   \includegraphics[width=1\textwidth]{immagini/capitolo2/snyk_output_fmt.png}
   \caption{Formato di output di default di Snyk}
   \label{fig:snyk_output_fmt}
\end{figure}

Inoltre, vengono mostrati anche:
\begin{itemize}
   \item Il numero totale di vulnerabilità rilevate.
   \item \textbf{Consigli per la base image}: Snyk propone una base image successiva o alternativa, per notificare l'utente che aggiornando la base image alcune delle vulnerabilità rilevate siano state risolte (Figura \ref{fig:snyk_altn_imgs}).
         \begin{figure}[H]
            \centering
            \includegraphics[width=1\textwidth]{immagini/capitolo2/snyk_altn_images.png}
            \caption{Snyk: Consigli per una base image alternativa.}
            \label{fig:snyk_altn_imgs}
         \end{figure}
\end{itemize}

Inoltre, Snyk permette di generare report nei seguenti formati:
\begin{itemize}
   \item \textbf{JSON}
   \item \textbf{SARIF}
\end{itemize}

Differentemente da Trivy, Snyk non permette di generare report in formati tabellari o personalizzati.

\section{Scansione di configurazioni IaC}
Un'altra funzionalità offerta da entrambi i tool è relativa alla scansione di configurazioni IaC, al fine di rilevare problemi di sicurezza o cattive pratiche di configurazione. Per testare tale funzionalità, si è utilizzato un campione di file di configurazione di Terraform, appropriatamente configurato con una vulnerabilità di esempio nota e ben documentata. Il codice in questione è il seguente:
\begin{lstlisting}
   provider "aws" {
      region = "us-east-1"
    }
    
    resource "aws_s3_bucket" "bucket_insecure" {
      bucket = "my-insecure-bucket"
      acl    = "public-read"
    
      tags = {
        Name        = "Insecure Bucket"
        Environment = "Test"
      }
    }
    
\end{lstlisting}


La vulnerabilità in questione è rappresentata dalla configurazione errata delle politiche di accesso su un bucket S3 di AWS, il quale è stato impostato per permettere l'accesso in lettura al pubblico di tutto il bucket (public-read). Tale configurazione espone i dati contenuti nel bucket a potenziali accessi non autorizzati, rappresentando un rischio significativo per la sicurezza dei dati.
Eseguendo la scansione del codice soprastante con entrambi gli strumenti, sono state rilevate le seguenti vulnerabilità:

\begin{center}
   \begin{tabularx}{0.8\textwidth}{|X|X|X|X|X|}
      \hline
      \textbf{Strumento} & \textbf{Critical} & \textbf{High} & \textbf{Medium} & \textbf{Low} \\
      \hline
      Snyk               & 0                 & 0             & 1               & 3            \\
      \hline
      Trivy              & 0                 & 7             & 1               & 2            \\
      \hline
   \end{tabularx}
\end{center}

Tra le vulnerabilità rilevate, in entrambi i casi

\section{Differenze di funzionalità}
Durante il testing, sono state rilevate le seguenti funzionalità uniche per ciascuno strumento:
\subsection{Trivy: scansione di cluster Kubernetes}
Trivy offre la possibilità di eseguire la scansione di un intero cluster Kubernetes, identificando e classificando le vulnerabilità presenti nei pod, nei deployment e nei servizi. Questa funzionalità è particolarmente utile per i team di sicurezza e per gli amministratori di sistema, che possono utilizzare Trivy per identificare e mitigare le vulnerabilità in modo proattivo, prima che possano essere sfruttate da attaccanti. Durante il testing, si è verificato che Trivy è in grado di eseguire la scansione di un cluster Kubernetes in modo rapido ed efficiente, fornendo un report dettagliato delle vulnerabilità rilevate e delle azioni consigliate per mitigarle.
\subsection{Trivy: Scansione delle Configurazioni Infrastructure as Code (IaC)}
Trivy offre la possibilità di eseguire la scansione delle configurazioni Infrastructure as Code (IaC), identificando e classificando le vulnerabilità presenti nei file di configurazione di Terraform, CloudFormation e altri strumenti di automazione dell'infrastruttura. Questa funzionalità è particolarmente utile per i team di sviluppo e per gli amministratori di sistema, che possono utilizzare Trivy per identificare e mitigare le vulnerabilità nelle configurazioni IaC, prima che possano essere sfruttate da attaccanti. Durante il testing, si è verificato che Trivy è in grado di eseguire la scansione delle configurazioni IaC in modo rapido ed efficiente, fornendo un report dettagliato delle vulnerabilità rilevate e delle azioni consigliate per mitigarle.

Per testare tale funzionalità, si è utilizzato un campione di file di configurazione di Terraform, appropriamente configurato con una vulnerabilità di esempio nota e ben documentata. Questa vulnerabilità è rappresentata dalla configurazione errata delle politiche di accesso su un bucket S3 di AWS, il quale è stato impostato per permettere l'accesso in lettura al pubblico (public-read). Tale configurazione espone i dati contenuti nel bucket a potenziali accessi non autorizzati, rappresentando un rischio significativo per la sicurezza dei dati.
Come è possibile osservare dalla figura \ref{fig:trivy_iac}, Trivy è stato in grado di identificare e classificare correttamente la vulnerabilità presente nella configurazione di Terraform, fornendo un report dettagliato delle azioni consigliate per mitigarla.
\begin{figure}[H]
   \centering
   \includegraphics[width=0.6\textwidth]{immagini/capitolo2/trivy_iac.png}
   \caption{Risultato della scansione delle configurazioni IaC con Trivy}
   \label{fig:trivy_iac}
\end{figure}
\subsection{Trivy:}

\subsection{Snyk: Monitoraggio Continuo}
Snyk offre un monitoraggio continuo delle applicazioni e delle dipendenze, inviando notifiche in tempo reale in caso di nuove vulnerabilità che influenzano il codice già in uso. Questo assicura che i team possano reagire rapidamente a nuove minacce. Durante il testing, è stato possibile osservare il funzionamento di questa funzione, avendo ricevuto e-mail contenenti nuove vulnerabilità pubblicate solamente giorni prima, come riportato in figura \ref{fig:snyk_email}.

\begin{figure}[H]
   \centering
   \includegraphics[width=0.8\textwidth]{immagini/capitolo2/snyk_email.png}
   \caption{E-mail ricevuta da Snyk contenente notifica di rilevazione di nuove vulnerabilità}
   \label{fig:snyk_email}
\end{figure}

Dopo la ricezione della e-mail, è stata subito eseguita una scansione con Trivy, che ha confermato la presenza delle vulnerabilità segnalate da Snyk. Questo conferma che entrambi i database utilizzati da Trivy e Snyk sono aggiornati tempestivamente.