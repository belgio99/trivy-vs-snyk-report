\chapter{Introduzione}
In un'epoca in cui il software permea ogni aspetto della nostra vita quotidiana, la sicurezza informatica è diventata una pietra angolare nello sviluppo e nel deployment delle applicazioni. La continua espansione dell'utilizzo dei container e delle microservizi ha portato alla necessità di strumenti sofisticati capaci di identificare e mitigare le vulnerabilità in modo efficace ed efficiente. Questo report si propone di esplorare e confrontare due dei più rilevanti strumenti nel panorama della sicurezza informatica: Trivy e Snyk. Entrambi gli strumenti hanno guadagnato notorietà per la loro capacità di fornire analisi dettagliate e soluzioni alle vulnerabilità di sicurezza in applicazioni e container, ma presentano approcci, caratteristiche e punti di forza distinti.

\section{Trivy}

Trivy, sviluppato da Aqua Security, è un scanner di vulnerabilità semplice e completo, che si distingue per la sua facilità d'uso e la capacità di integrarsi senza soluzione di continuità in vari ambienti di sviluppo e pipeline CI/CD. La sua progettazione si concentra sulla velocità e sull'efficacia della scansione di immagini di container, repository Git, filesystem e configurazioni Infrastructure as Code (IaC), rendendolo uno strumento versatile per gli sviluppatori e i team di sicurezza.

\section{Snyk}

Snyk si posiziona come una soluzione SaaS di sicurezza per lo sviluppo software che enfatizza la collaborazione tra sviluppatori e professionisti della sicurezza. Offre una vasta gamma di funzionalità che vanno oltre la semplice scansione delle vulnerabilità, inclusa la gestione delle dipendenze, la correzione automatica e l'integrazione profonda con ambienti di sviluppo, sistemi di gestione del codice sorgente e pipeline di CI/CD. Snyk mira a dotare i team di strumenti proattivi per affrontare le vulnerabilità all'interno del loro codice, delle dipendenze open source, dei container e delle configurazioni IaC.

Il confronto tra Trivy e Snyk richiede un'analisi approfondita delle loro capacità tecniche, facilità d'uso, integrazione con gli ambienti di sviluppo esistenti, e l'impatto sul flusso di lavoro di sviluppo e sicurezza. Questo report valuterà questi aspetti attraverso una metodologia dettagliata che include test empirici, interviste con gli utenti e l'analisi della documentazione ufficiale e delle recensioni della comunità. L'obiettivo è fornire una panoramica esauriente che aiuti gli sviluppatori, i team di sicurezza e le organizzazioni a prendere decisioni informate sulla scelta dello strumento più adatto alle loro esigenze specifiche nel contesto della sicurezza delle applicazioni e dei container.